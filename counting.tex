

\documentclass[11pt,a4paper,titlepage,oneside,openany]{article}

\pagestyle{plain}
%\renewcommand{\baselinestretch}{1.7}

\usepackage{setspace}
%\singlespacing
\onehalfspacing
%\doublespacing
%\setstretch{1.1}

\usepackage{amsmath}
\usepackage{amssymb}
\usepackage{amsthm}
\usepackage{framed}
\usepackage{multicol}

\usepackage[margin=3cm]{geometry}
\usepackage{graphicx,psfrag}%\usepackage{hyperref}
\usepackage[small]{caption}
\usepackage{subfig}

\usepackage{algorithm}
\usepackage{algorithmic}
\newcommand{\theHalgorithm}{\arabic{algorithm}}

\usepackage{varioref} %NB: FIGURE LABELS MUST ALWAYS COME DIRECTLY AFTER CAPTION!!!
%\newcommand{\vref}{\ref}

\usepackage{index}
\makeindex
\newindex{sym}{adx}{and}{Symbol Index}
%\newcommand{\symindex}{\index[sym]}
%\newcommand{\symindex}[1]{\index[sym]{#1}\hfill}
\newcommand{\symindex}[1]{\index[sym]{#1}}

%\usepackage[breaklinks,dvips]{hyperref}%Always put after varioref, or you'll get nested section headings
%Make sure this is after index package too!
%\hypersetup{colorlinks=false,breaklinks=true}
%\hypersetup{colorlinks=false,breaklinks=true,pdfborder={0 0 0.15}}


%\usepackage{breakurl}

\graphicspath{{./images/}}

\usepackage[subfigure]{tocloft}%For table of contents
\setlength{\cftfignumwidth}{3em}

\input{longdiv}
\usepackage{wrapfig}


%\usepackage{index}
%\makeindex
%\usepackage{makeidx}

%\usepackage{lscape}
\usepackage{pdflscape}
\usepackage{multicol}

\usepackage[utf8]{inputenc}

%\usepackage{fullpage}

%Compulsory packages for the PhD in UL:
%\usepackage{UL Thesis}
\usepackage{natbib}

%\numberwithin{equation}{section}
\numberwithin{equation}{section}
\numberwithin{algorithm}{section}
\numberwithin{figure}{section}
\numberwithin{table}{section}
%\newcommand{\vec}[1]{\ensuremath{\math{#1}}}

%\linespread{1.6} %for double line spacing

\usepackage{afterpage}%fingers crossed

\newtheorem{thm}{Theorem}[section]
\newtheorem{defin}{Definition}[section]
\newtheorem{cor}[thm]{Corollary}
\newtheorem{lem}[thm]{Lemma}

%\newcommand{\dbar}{{\mkern+3mu\mathchar'26\mkern-12mu d}}
\newcommand{\dbar}{{\mkern+3mu\mathchar'26\mkern-12mud}}

\newcommand{\bbSigma}{{\mkern+8mu\mathsf{\Sigma}\mkern-9mu{\Sigma}}}
\newcommand{\thrfor}{{\Rightarrow}}

\newcommand{\mb}{\mathbb}
\newcommand{\bx}{\vec{x}}
\newcommand{\bxi}{\boldsymbol{\xi}}
\newcommand{\bdeta}{\boldsymbol{\eta}}
\newcommand{\bldeta}{\boldsymbol{\eta}}
\newcommand{\bgamma}{\boldsymbol{\gamma}}
\newcommand{\bTheta}{\boldsymbol{\Theta}}
\newcommand{\balpha}{\boldsymbol{\alpha}}
\newcommand{\bmu}{\boldsymbol{\mu}}
\newcommand{\bnu}{\boldsymbol{\nu}}
\newcommand{\bsigma}{\boldsymbol{\sigma}}
\newcommand{\bdiff}{\boldsymbol{\partial}}

\newcommand{\tomega}{\widetilde{\omega}}
\newcommand{\tbdeta}{\widetilde{\bdeta}}
\newcommand{\tbxi}{\widetilde{\bxi}}



\newcommand{\wv}{\vec{w}}

\newcommand{\ie}{i.e. }
\newcommand{\eg}{e.g. }
\newcommand{\etc}{etc}

\newcommand{\viceversa}{vice versa}
\newcommand{\FT}{\mathcal{F}}
\newcommand{\IFT}{\mathcal{F}^{-1}}
%\renewcommand{\vec}[1]{\boldsymbol{#1}}
\renewcommand{\vec}[1]{\mathbf{#1}}
\newcommand{\anged}[1]{\langle #1 \rangle}
\newcommand{\grv}[1]{\grave{#1}}
\newcommand{\asinh}{\sinh^{-1}}

\newcommand{\sgn}{\text{sgn}}
\newcommand{\morm}[1]{|\det #1 |}

\newcommand{\galpha}{\grv{\alpha}}
\newcommand{\gbeta}{\grv{\beta}}
%\newcommand{\rnlessO}{\mb{R}^n \setminus \vec{0}}
\usepackage{listings}

\interfootnotelinepenalty=10000

\newcommand{\sectionline}{%
  \nointerlineskip \vspace{\baselineskip}%
  \hspace{\fill}\rule{0.5\linewidth}{.7pt}\hspace{\fill}%
  \par\nointerlineskip \vspace{\baselineskip}
}

\renewcommand{\labelenumii}{\roman{enumii})}

\begin{document}


\section{Pascal's Identity}
Pascal's identity, first derived by Blaise Pascal in 19th century, states that the number of ways to choose k elements from n elements is equal to the summation of number of ways to choose (k-1) elements from (n-1) elements and the number of ways to choose elements from n-1 elements.

Mathematically, for any positive integers k and n: \[ nCk=n−1Ck−1+n−1CknCk=n−1Ck−1+n−1Ck \]
Proof −

\begin{eqnarray}
{ n\choose k} n−1Ck−1+n−1Ck
n−1Ck−1+n−1Ck\\
=(n−1)!(k−1)!(n−k)!+(n−1)!k!(n−k−1)!=(n−1)!(k−1)!(n−k)!+(n−1)!k!(n−k−1)!\\
=(n−1)!(kk!(n−k)!+n−kk!(n−k)!)=(n−1)!(kk!(n−k)!+n−kk!(n−k)!)\\
=(n−1)!nk!(n−k)!=(n−1)!nk!(n−k)!\\
=n!k!(n−k)!=n!k!(n−k)!\\
=nCk=nCk\\
\end{eqnarray}
\subsection{Pigeonhole Principle}
In 1834, German mathematician, Peter Gustav Lejeune Dirichlet, stated a principle which he called the drawer principle. Now, it is known as the pigeonhole principle.

Pigeonhole Principle states that if there are fewer pigeon holes than total number of pigeons and each pigeon is put in a pigeon hole, then there must be at least one pigeon hole with more than one pigeon. If n pigeons are put into m pigeonholes where n > m, there's a hole with more than one pigeon.

\subsection{Examples}
Ten men are in a room and they are taking part in handshakes. If each person shakes hands at least once and no man shakes the same man’s hand more than once then two men took part in the same number of handshakes.

There must be at least two people in a class of 30 whose names start with the same alphabet.

\newpageThe
\section{The Inclusion-Exclusion principle}
The Inclusion-exclusion principle computes the cardinal number of the union of multiple non-disjoint sets. For two sets A and B, the principle states −

\[|A∪B|=|A|+|B|−|A∩B||A∪B|=|A|+|B|−|A∩B|\]
For three sets A, B and C, the principle states −

\[|A∪B∪C|=|A|+|B|+|C|−|A∩B|−|A∩C|−|B∩C|+|A∩B∩C||A∪B∪C|=|A|+|B|+|C|−|A∩B|−|A∩C|−|B∩C|+|A∩B∩C|\]
The generalized formula -

\[ |⋃ni=1Ai|=∑1≤i<j<k≤n|Ai∩Aj|+∑1≤i<j<k≤n|Ai∩Aj∩Ak|−⋯+(−1)π−1|A1∩⋯∩A2||⋃i=1nAi|=∑1≤i<j<k≤n|Ai∩Aj|+∑1≤i<j<k≤n|Ai∩Aj∩Ak|−⋯+(−1)π−1|A1∩⋯∩A2|
\]
Problem 1

How many integers from 1 to 50 are multiples of 2 or 3 but not both?

\subsection{Solution}

\begin{itemize}
    \item From 1 to 100, there are 50/2=2550/2=25 numbers which are multiples of 2.

\item There are 50/3=1650/3=16 numbers which are multiples of 3.

\item There are 50/6=850/6=8 numbers which are multiples of both 2 and 3.

\item So, |A|=25|A|=25, |B|=16|B|=16 and |A∩B|=8|A∩B|=8.

\item |A∪B|=|A|+|B|−|A∩B|=25+16−8=33|A∪B|=|A|+|B|−|A∩B|=25+16−8=33
\end{itemize}

Problem 2

In a group of 50 students 24 like cold drinks and 36 like hot drinks and each student likes at least one of the two drinks. How many like both coffee and tea?

Solution

Let X be the set of students who like cold drinks and Y be the set of people who like hot drinks.

So, |X∪Y|=50|X∪Y|=50, |X|=24|X|=24, |Y|=36|Y|=36
|X∩Y|=|X|+|Y|−|X∪Y|=24+36−50=60−50=10|X∩Y|=|X|+|Y|−|X∪Y|=24+36−50=60−50=10
Hence, there are 10 students who like both tea and coffee.

% - https://en.wikipedia.org/wiki/Inclusion%E2%80%93exclusion_principle#/media/File:Inclusion-exclusion.svg

\end{document}
