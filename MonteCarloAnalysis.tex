\subsection{Introduction}
Having been named after the principality famous for its casinos, the term Monte Carlo Analysis conjures images of an intricate strategy aimed at maximizing one's earnings in a casino game.

However, Monte Carlo Analysis refers to a technique in project management where a manager computes and calculates the total project cost and the project schedule many times.

This is done using a set of input values that have been selected after careful deliberation of probability distributions or potential costs or potential durations.

\subsection{Importance of the Monte Carlo Analysis}
The Monte Carlo Analysis is important in project management as it allows a project manager to calculate a probable total cost of a project as well as to find a range or a potential date of completion for the project.

Since a Monte Carlo Analysis uses quantified data, this allows project managers to better communicate with senior management, especially when the latter is pushing for impractical project completion dates or unrealistic project costs.

Also, this type of an analysis allows the project managers to quantify perils and ambiguities in project schedules.

A Simple Example of the Monte Carlo Analysis
A project manager creates three estimates for the duration of the project: one being the most likely duration, one the worst case scenario and the other being the best case scenario. For each estimate, the project manager consigns the probability of occurrence.

The project is one that involves three tasks:
\begin{enumerate}
\item The first task is likely to take three days (70% probability), but it can also be completed in two days or even four days. The probability of it taking two days to complete is 10% and the probability of it taking four days to finish is 20%.

\item The second task has a 60% probability of taking six days to finish, a 20% probability each of being completed in five days or eight days.
\item The final task has an 80% probability of being completed in four days, 5% probability of being completed in three days and a 15% probability of being completed in five days.
\end{itemize}


Using the Monte Carlo Analysis, a series of simulations are done on the project probabilities. The simulation is to run for a thousand odd times, and for each simulation, an end date is noted.

Once the Monte Carlo Analysis is completed, there would be no single project completion date. Instead the project manager has a probability curve depicting the likely dates of completion and the probability of attaining each.

Using this probability curve, the project manager informs the senior management of the expected date of completion. The project manager would choose the date with a 90% chance of attaining it.

Therefore, it could be said that using the Monte Carlo Analysis, the project has a 90% chance of being completed in X number of days.

Similarly, a project manager can adjudge the estimated budget for a project using probabilities to simulate different end results and in turn use the findings in a probability curve.

\subsection{How is the Monte Carlo Analysis Carried Out?}
The above example was one that contained a mere three tasks. In reality, such projects contain hundreds if not thousands of tasks.

Using the Monte Carlo Analysis, a project manager is able to derive a probability curve to show the ambiguity surrounding the duration and the costs surrounding these hundreds or thousands of tasks.

Conducting simulations involving hundreds or thousands of tasks is a tedious job to be done manually.

Today there is project management scheduling software that can conduct thousands of simulations and offer the project manager different end results in a probability curve.

\subsection{The Different Types of Probability Distributions/Curves}
A Monte Carlo Analysis shows the risk analysis involved in a project through a probability distribution that is a model of possible values.

Some of the commonly used probability distributions or curves for Monte Carlo Analysis include:

The Normal or Bell Curve - In this type of probability curve, the values in the middle are the likeliest to occur.

The Lognormal Curve - Here values are skewed. A Monte Carlo Analysis gives this type of probability distribution for project management in the real estate industry or oil industry.

The Uniform Curve - All instances have an equal chance of occurring. This type of probability distribution is common with manufacturing costs and future sales revenues for a new product.

The Triangular Curve - The project manager enters the minimum, maximum or most likely values. The probability curve, a triangular one, will display values around the most likely option.

Conclusion
The Monte Carlo Analysis is an important method adopted by managers to calculate the many possible project completion dates and the most likely budget required for the project.

Using the information gathered through the Monte Carlo Analysis, project managers are able to give senior management the statistical evidence for the time required to complete a project as well as propose a suitable budget.
